\documentclass{article}
\usepackage{mathtools}
\usepackage[english, russian]{babel}

\title{Lambda. Homework 3}
\author{Kargin Gleb}
\date{}

\begin{document}

    \maketitle


    \section{First Task}
    $
    ((\lambda a.(\lambda b.b b) (\lambda b.b b)) b) ((\lambda c.(c b)) (\lambda a.a)) \xRightarrow[\beta]{} ((\lambda b.b b) (\lambda b.b b)) ((\lambda c.(c b)) (\lambda a.a))
    $

    Заметим, что в результате нормальной стратегии не происходит сокращение терма. Таким образом, можно сделать вывод, что нормальной формы нет.

    Продолжим редуцировать:

    $
    ((\lambda b.b b) (\lambda b.b b)) ((\lambda c.(c b)) (\lambda a.a))
    \xRightarrow[\beta]{} ((\lambda b.b b) (\lambda b.b b)) ((\lambda a.a) b) \xRightarrow[\beta]{} (\lambda b.b b) (\lambda b.b b) b
    \xRightarrow[\beta]{} \Omega b
    $


    \section{Second Task}
    \sloppy
    $
    S K K = (\lambda x y z.x z (y z)) (\lambda x y.x) (\lambda x y.x) \xRightarrow[\beta]{} (\lambda y z.(\lambda x y.x) z (y z)) (\lambda x y.x) \xRightarrow[\beta]{} (\lambda z. (\lambda x y.x) z ((\lambda x y.x)z))
    \xRightarrow[\beta]{} (\lambda z. (\lambda y.z) ((\lambda x y.x)z))
    \xRightarrow[\beta]{} (\lambda z. (\lambda ((\lambda x y.x)z).z) )
    \xRightarrow[\beta]{} (\lambda z. z) = I
    $

\end{document}
